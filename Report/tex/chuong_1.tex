\chapter{Cơ sở lý thuyết}
\section{Html}
Html là ngôn ngữ được dùng để định nghĩa cấu trúc của một trang web, được ra đời từ những năm 1990, cho tới nay HTML đã trải qua năm phiên bản, tại phiên bản hiện tại (HTML5) HTML được bổ sung thêm nhiều tính năng mới, như WebRTC, trình chơi video, vẽ canvas, service workers, ... giúp xây dựng các ứng dụng web đa dạng hơn trước, ngoài ra HTML còn cung cấp một số thẻ semantic giúp bọ của Google, hay những công cụ tìm kiếm khác dễ dàng đọc cấu trúc của website hơn. Điều này mang tới lợi ích cho việc SEO nhưng không bắt buộc lập trình viên phải sử dụng.
\section{Css}
Css là ngôn ngữ dùng để thêm các thuộc tính hiển thị cho trang web, giúp trang web Trông đẹp mắt và sống động hơn. Bằng cách định nghĩa thuộc tính cho các thẻ trong HTML, trình duyệt có thể hiểu và hiển thị trang web tới người dùng.
\section{Js}
Thay vì chỉ làm website trông sống động hơn, JS giúp trang web thực sự linh động, giúp người dùng có thể tương tác với trang web giống như một phần mềm hay ứng dụng bản địa mà không cần phải làm mới một cách thủ công như trước. Sự ra đời của JS giúp website tương tác với người dùng tốt hơn.
 \section{Bootstrap}
 Bootstrap là một framework css, được Twitter chống lưng, nói đơn giản thì Bootstrap chính là mã css được định nghĩa sẵn nhằm giúp lập trình viên nhanh chóng tạo ra các ứng dụng web mà không phải lo lắng quá nhiều tới việc căn chỉnh các thành phần trong css, ngoài ra Bootstrap cũng đảm bảo việc tương thích giữa các trình duyệt, giúp lập trình viên chỉ cần tập trung vào việc lập trình mà không mất nhiều công sức chỉnh sửa và test trên các thiết bị, trình duyệt khác nhau. Hiện tại Bootstrap đã ra mắt tới phiên bản thứ 4.
 \section{Php}
 Php là một ngôn ngữ chạy ở phía server, hay backend. Tuy có khá nhiều người nhận xét là Php kém bảo mật, xong đây là một ngôn ngữ khá dễ học, cộng đồng phát triển đông đảo, nhiều framework mã nguồn mở khác cũng được xây dựng từ Php có thể kể đến như Laravel, Wordpress, CodeIgniter, ..... ngoài ra Php cũng có khả năng tương thích ngược khá tốt, tốc độ thì ngày càng được cải thiện, hiện tại vẫn có tới 80\% website được xây dựng bằng Php.
